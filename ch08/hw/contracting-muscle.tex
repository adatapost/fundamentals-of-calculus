The figure shows the tension (force) of which a muscle is capable.
The variable $x$ is defined as the contraction of the muscle from its maximum length $L$, so that at
$x=0$ the muscle has length $L$, and at $x=L$ the muscle would theoretically have zero length.
In reality, the muscle can only contract to $x=cL$, where $c$ is less than 1.
When the muscle is extended to its maximum length, at $x=0$, it is capable of the greatest tension, $T_\zu{o}$.
As the muscle contracts, however, it becomes weaker. There is a nearly linear
decrease, which would theoretically extrapolate to zero at $x=L$.
(a)  Infer the units of $c$ and $T_\zu{o}$.
(b) Find the maximum work the muscle can do in one contraction, in terms of $c$, $L$, and $T_\zu{o}$.
(c) Show that your answer to part b has the right units.
(d) Show that your answer to part b has the right behavior when $c=0$ and when $c=1$.\answercheck
