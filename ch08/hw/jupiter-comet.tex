In July 1994, Comet Shoemaker-Levy 9, which had previously broken up into pieces,
collided with the planet Jupiter.
The figure shows discolorations left in the jovian atmosphere where the impacts
had occurred. The diameter of each bruise is on the same order of magnitude as
the size of the planet earth. These were hard hits. The energy came from the work
done by the sun's gravity on the comet as it fell inward from the Oort Cloud, a
hypothesized outer region of the solar system. Let $x$ be the comet's position
relative to the sun, and assume that the comet falls in from the negative $x$ direction,
i.e., from the side of the sun that we would visualize as the left-hand side of the
number line. 
The force of the sun's gravity on the comet is given by Newton's law of gravity,
$F = GMm/x^2$,
where $M$ is the mass of the sun, $m$ is the mass of the comet,
$G$ is a universal constant, and the plus sign indicates that the force is
to the right, i.e., toward the sun.\\
(a) Infer the units of $G$.
(b) Find the work done on the comet as it falls from $x=-a$ to $x=-b$, where $a$ is the distance
from the sun to the Oort cloud, $b$ is the distance from the sun to Jupiter, and both $a$ and
$b$ are positive.
(c) Check that the units of your answer to part b make sense.\answercheck
