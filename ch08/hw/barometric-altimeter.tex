A barometric altimeter is a device that uses a measurement of air pressure $p$ to determine
altitude $y$. Let the density of air be $\rho$ (Greek letter ``rho,'' the equivalent of Latin ``r,'' not
``p''), and the strength of the earth's gravitational field $g$. If $\rho$ is constant, then the
difference in pressure between two heights is given by
\begin{equation*}
  \Delta p = \rho g \Delta y \qquad .
\end{equation*}
Mountaineers and airplane pilots often traverse large enough changes in altitude that it is not
a good approximation to take $\rho$ as being constant; the air is less dense higher up.
Use one of the methods of section \ref{subsec:to-differentiate-or-to-integrate},
p.~\pageref{subsec:to-differentiate-or-to-integrate}, to generalize the equation.
