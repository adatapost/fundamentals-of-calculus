A pendulum is pulled back through an angle $\theta$ and then released.
It then swings from $\theta$ to $-\theta$ and back to $\theta$ again;
this is considered one complete oscillation.
The time it takes to carry out this oscillation is called the period,
$T$. If the pendulum is hung on a stiff rod rather than with a string,
then $\theta$ can be as big as $180\degunit$; you will find it helpful
to consider what happens in the extreme case where $\theta$ \emph{equals}
$180\degunit$.
As in the examples
in section \ref{subsec:sketch-without-equation}, sketch the function
$T(\theta)$ without knowing its equation.
