If we want to pump air or water through a pipe, common sense tells us that it will be easier
to move a larger quantity more quickly through a fatter pipe. Quantitatively, we can define
the resistance, $R$, which is the ratio of the pressure difference produced by the pump to the
rate of flow. A fatter pipe will have a lower resistance. Two pipes can be used in parallel,
for instance when you turn on the water both in the kitchen and in the bathroom, and in this
situation, the two pipes let more water flow than either would have let flow by itself, which
tells us that they act like a single pipe with some lower resistance. The equation for their
combined resistance is $R=1/(1/R_1+1/R_2)$.\hwendpart
(a) Analyze the case where one resistance is fixed
at some finite value, while the other approaches infinity. Give a physical
interpretation.\hwendpart
(b) Likewise, discuss the case where
one is finite, but the other becomes very small.\hwendpart
