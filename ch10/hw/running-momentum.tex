Scientists in Daniel Lieberman's Skeletal Biology Lab at Harvard
specialize in measuring the forces that act on a runner's body,
which may help to improve coaching, reduce injuries, or provide
scientific evidence about whether barefoot running is healthier
than using running shoes.
The graph in the figure shows a typical
result for the \emph{vertical} force as a function of time that acts between
the runner's foot and a treadmill, for one portion of a stride
cycle. 

The initial time $t=0$ is the one when the vertical force is at
its greatest, shown in the drawing. At this time, the runner's body is
about as low as it will get, and the vertical momentum is approximately zero.

The end of the graph, where the
force goes to zero, is the time at which the runner's back toe leaves
the ground and he becomes airborne for a fraction of a second.

The
graph looks like a parabola, so let's model it as one, $F=b(1-t^2/\tau^2)-w$,
where $\tau$ is the time at which the graph ends, and the $-w$ term accounts
for gravity. (a) Infer the units of the constants $b$, $\tau$, and $w$.
(b) Find the runner's vertical momentum at $t=\tau$, i.e., the momentum with which
he takes off into the air.
(c) Check that your answer to part b has units that make sense.\answercheck
