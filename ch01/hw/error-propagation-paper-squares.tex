% RA, p. 35, #1 (group project)
Joe sells square sheets of gold foil. Since gold is expensive, the sheets
are sold by area $a$. If the area is too small, the customer gets upset, but
if the area is too high, Joe is losing money. Therefore he wants to make
sure that the area doesn't differ from $a$ by more than $\Delta a$.
In his shop, Joe marks off squares of length $x$.\\
(a) No measurement is perfectly exact. By what amount $\Delta x$ can his
length measurement be off if the resulting error in the area is
to be no more than $\Delta a$? Use the approximation method
described in section \ref{subsec:approximating-changes} on 
p.~\pageref{subsec:approximating-changes}.\answercheck\hwendpart
(b)  Check that your answer has the right units, as in example 
\ref{eg:pest} on page \pageref{eg:pest} and section \ref{sec:more-about-units} on p.~\pageref{sec:more-about-units}.\hwendpart
(c) If the desired area is $a=4.000\ \munit^2$, and
the maximum allowable error in area is $0.001\ \munit^2$,
what is the biggest error Joe can afford to make when he
marks off the length $x$? Express your result using an appropriate unit or in scientific notation, not
as an awkward decimal with a string of zeroes.\answercheck
