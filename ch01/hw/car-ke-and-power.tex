Kinetic energy is a measure of an object's quantity of motion; when you buy gasoline,  the
energy you're paying for will be converted into the car's kinetic energy (actually only some of
it, since the engine isn't perfectly efficient). The kinetic energy of an object with mass
$m$ and velocity $v$ is given by $K=(1/2)mv^2$.\\
(a) As described in box \figref{units} on p.~\pageref{fig:units}, infer the SI units of kinetic energy.\\
(b) For a car accelerating at a steady rate, with
$v=at$, find the rate $\der K/\der t$ at which the engine is required to put out kinetic energy.
$\der K/\der t$, with units of energy over time, is known as the \emph{power}.
Hint: We're differentiating with respect to $t$, and the thing being squared is not
just $t$, so this is not a form that you know how to differentiate. Use algebra
to convert it into a form that you do know how to handle.\answercheck\hwendpart
(c) Check that your answer has the right units, as in example \ref{eg:pest} on page \pageref{eg:pest}
and section \ref{sec:more-about-units} on p.~\pageref{sec:more-about-units}.
