A seller offers something at a unit price $P$, and the quantity of units
sold is $Q$. Ordinarily, we expect that $P$ and $Q$ would be related in some
way that could be expressed by a graph, but there's no obvious way to decide
which variable, $P$ or $Q$, should be on which axis. The cause-and-effect relationship
isn't clearly one way or the other: a change in price could cause a change in demand,
but a change in demand could also prompt the seller to change the price. 
The graph is called the \emph{demand curve}.\index{demand curve}

For some
unusual goods, the demand is insensitive to the price. For example, the drug
Soliris treats a genetic disease so rare that only about 8,000 people in the U.S.~have it.
The price $P$ is about \$400,000 per patient per year. Since the benefits
of treatment for these people are so great, and the cost is paid for by government
or private insurers, changing $P$ would not change $Q$. (a) How would this example look
on a graph if we put $P$ on the $y$ axis and $Q$ on the $x$ axis? What if we did
it the other way around? (b) In each case, discuss whether the graph is a function.
(c) In each case, what can you say about
the derivative based on the
the informal definition given in section
\ref{subsec:informal-derivative}?
