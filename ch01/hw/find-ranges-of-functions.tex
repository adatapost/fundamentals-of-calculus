Recall that the \emph{range} of a function is the set of possible values its
output can have.\index{range of a function}\index{function!range of}
Find the ranges of the following functions.
\begin{align*}
  f(x) &= 2x^2+3 \\
  g(x) &= -2x^2+4x \\
  h(x) &= 4x +x^2\\
  k(x) &= 1/(1+x^2)\\
  \ell(x) &= 1/(3+2x+x^2)\\
  m(x) &= 4\sin x + \sin^2 x 
\end{align*}
(For $m$, if you've forgotten your trig you may wish to review
from section \ref{sec:trig}, p.~\pageref{sec:trig}. It is possible
to do this problem without knowing how to differentiate the sine
function.)

You will find it convenient to express some of your answers using
notations such as $[17,\infty)$, which is a standard way of extending
the normal notation for finite intervals (p.~\pageref{interval-notation})
to describe infinite ones. This example means, as you'd
imagine, the set $\{u|u\ge 17\}$. Although $\infty$ isn't
a real number, the notation gets the idea across. The use of the $)$
rather than a $]$ is to show that there isn't a member of the set
whose value is infinite.\index{interval!infinite!notation for}

Although you may be able to guess some of the answers by constructing
a graph, that does not constitute a proof of the exact result.
