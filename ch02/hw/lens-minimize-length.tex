A camera takes light from an object and forms an image on the
film or computer chip at the back of the camera inside its body. Let $u$ be the distance
from the object to the lens, and $v$ the distance from the lens
to the image. These distances are related by the equation
\begin{equation*}
  \frac{1}{f} = \frac{1}{u}+\frac{1}{v}\eqquad,
\end{equation*}
where $f$ is a fixed property of the lens, called its focal length.
When we want to focus on an object at a particular distance,
we have to move the lens in or out so that $u$ and $v$ fulfill
this equation; in an autofocus camera this is done automatically
by a small motor. Let
\begin{equation*}
  L = u+v
\end{equation*}
be the distance from the object to the back of the camera's body,
and suppose that we want to take a picture of an object as nearby
as possible, in the sense of minimizing $L$.\hwendpart
(a) Solve the first equation for $v$, and substitute into the
second equation to eliminate $v$, thereby expressing $L$ as a function
that depends only on the variable $u$ (and the constant $f$).\answercheck\hwendpart
(b) Find the value of $u$ that minimizes the function $L(u)$.\answercheck\hwendpart
(c) Find the minimum value of $L$.\answercheck\hwendpart
