A fancy factory can't produce anything if it has no workers
to keep it running, but on the other hand a big crowd of
workers standing around in a vacant lot also can't do anything.
Businesses need to balance their spending on labor $L$ and
the amount $E$ invested in capital equipment, such as machinery. In 1928, economists
Charles Cobb and Paul Douglas used macroeconomic data from
the U.S. to come up with the following model for production.
\begin{equation*}
  P = cL^\alpha E^{1-\alpha}
\end{equation*}
Here $P$ is the amount produced, and $c$ and $\alpha$ are constants.
Suppose that a business has a fixed amount of capital $T$, so that
\begin{equation*}
  L+E = T \qquad .
\end{equation*}
(a) Use the second equation to eliminate $E$, and find the optimal
fraction $L/T$ of capital that should be spent on labor.
(b) Show that your answer has the correct behavior in the special
cases $\alpha=0$, 1/2, and 1.\answercheck
