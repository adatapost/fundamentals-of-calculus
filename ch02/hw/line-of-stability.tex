An atomic nucleus is made out of protons and neutrons. The number of protons
is called $Z$ and the number of neutrons $N$. Figure \figref{hw-line-of-stability}
on p.~\pageref{fig:hw-line-of-stability} shows a chart of all of the nuclei that have
been observed and studied to date. Most of these are unstable: they undergo radioactive
decay in a certain amount of time, and therefore are not found in the earth's crust,
so they can only be produced artificially.

The stable nuclei are shown on the chart
as black squares, and we can see that they follow a certain curve. Unstable nuclei
that lie below and to the right of the line of stability have too many neutrons in
proportion to their protons, and they undergo a decay process in which a neutron
is converted to a proton, causing the nucleus to move one step diagonally on the
chart, as in the game of checkers. Similarly, nuclei with too few neutrons move
by diagonal steps down and to the right. Defining $A=N+Z$, these decay processes keep
$A$ constant.

In the \emph{liquid drop model}, the nucleus is treated as a continuous fluid with
certain properties such as surface tension. Since the fluid is continuous, we can
pretend that $N$ and $Z$ are capable of taking on any real-number values. (This is similar to
the water molecules in the reservoir on p.~\pageref{fig:reservoir}.) In this model,
a nucleus has a certain energy,
\begin{equation*}
  E = bZ^2A^{-1/3}+\frac{(A-2Z)^2}{A} \qquad ,
\end{equation*}
where $b\approx0.031$, and for simplicity we have left out
an over-all constant of proportionality with units of energy.
Let's consider $E$ as a function of $Z$, and $A$ as a constant.
Since radioactive decay requires the release of energy, and our radioactive decay
processes keep $A$ constant, a nucleus will be stable if it has the value
of $Z$ that minimizes the function $E(Z)$.

(a) Find this stable value of $Z$.\answercheck\hwendpart
(b) For light nuclei, we observe that the stable nuclei have about half protons and
half neutrons. Verify this from your answer to part a.\hwendpart
(c) The heaviest nucleus shown as a black square on the chart is a uranium nucleus
with $Z=92$ and $A=238$. Verify that your answer to part a passes close to this point.
% calc -e "Z=92; N=146; A=N+Z; b=.031; A/(2+.5bA^(2/3))"
%    91.7051812806395
