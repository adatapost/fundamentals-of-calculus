Suppose that we put a stick on a table and use a ruler
to measure its length $L$.
According to Einstein's theory of special relativity,
if the stick is instead in motion at speed $v$ relative to the ruler, then
we get a different, shorter length given by
\begin{equation*}
  M = L\sqrt{1-\frac{v^2}{c^2}}\eqquad,
\end{equation*}
where $c$ is the speed of light. We don't notice this effect in
everyday life because ordinary velocities are so small compared
to $c$.
(a) Calculate $\der M/\der v$, the rate at which the stick shortens
with increasing speed. (b) Check the units of your answer.
(c) Check that the sign of the result makes sense.
(d) Discuss the behavior of your result if $v=c$.\answercheck
