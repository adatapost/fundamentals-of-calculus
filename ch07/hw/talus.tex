The photo shows a common geological formation called \emph{talus}. Erosion causes
rock and sand to be washed down the gullies, where  over geological time
this debris piles up higher and higher
against the vertical cliff.
Suppose that the pile is
in the shape of half a cone, and that its volume grows at a
rate $R=\der V/\der t$.
The cone's slope $\alpha$ is fixed by the maximum steepness
for which friction is capable of keeping a rock from sliding down.
(a) Find the rate $\der h/\der t$ at which the height of the cone
grows, in terms of $R$, $\alpha$, and $h$.
(b) Check that your answer to part a has units that make sense.
(c) Check the dependence of your answer on the variable $R$.
That means that you should determine \emph{physically} whether increasing $R$ should increase
the result or decrease it, and then compare this to the \emph{mathematical}
behavior of your equation.
(d) Do the same for the variables $\alpha$ and $h$.
