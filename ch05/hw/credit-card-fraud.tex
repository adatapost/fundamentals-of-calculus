Credit card fraud creates costs (including both economic costs and
inconvenience) for businesses, credit card holders, and the credit
card companies. If the company institutes a particular measure to
prevent fraud, it may be able to eliminate some fraction of the fraud
that would otherwise have occurred. Putting some additional measure in place
may then eliminate some fraction of the remaining fraud, further reducing
the total amount. Let the amount the company spends on prevention be $p$.
For the reasons described above, it's reasonable to imagine that fraud
falls off exponentially as a function of $p$, so that the total cost
to the company is
\begin{equation*}
  C(p) = p+ae^{-bp}\eqquad.
\end{equation*}
Here $a$ and $b$ are constants, the first term represents the cost of
carrying out the fraud prevention, and the second term represents
the cost of the fraud that was not prevented.\\
(a) Find the value of $p$ that minimizes the cost.\answercheck\hwendpart
(b) Check that the units of your answer make sense.\hwendpart
(c) For what values of the parameters $a$ and
$b$ does your answer not produce a meaningful result? Check that this
makes sense.\hwendpart
(d) Suppose that legislation forces the credit card company to suffer
more of the consequences of the fraud, rather than making their customers
bear the brunt. What change does this imply in the parameters of the model?
Check that your answer to part a shows the right trend when this change
is applied.
