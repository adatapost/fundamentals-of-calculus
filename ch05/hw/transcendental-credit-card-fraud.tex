In problem \ref{hw:credit-card-fraud} on p.~\pageref{hw:credit-card-fraud},
we minimized a function that looked like
\begin{equation*}
  y = x+ae^{-bx} \qquad ,
\end{equation*}
where $x$, $a$, and $b$ were all positive. Suppose instead that
the function had been
\begin{equation*}
  y = x^2+ae^{-bx} \qquad ,
\end{equation*}
with the corresponding quantities still being positive.
Using the same technique to find its minimum, we obtain an equation
of a type called a transcendental equation, which cannot be solved
exactly for $x$ in terms of elementary functions. Use the intermediate value
theorem to prove that such a minimum nevertheless exists, as long as $a$ and
$b$ are both greater than zero.
% checked for a=1 and b=1:
% calc -e "x=.352; 2x-e^(-x)"
%    x = 0.352
%    7.1987802365947*10^-4
% calc -e "x=.352; x^2+e^(-x)"
%    0.827184121976341
% calc -e "x=.4; x^2+e^(-x)"
%    0.830320046035639
% calc -e "x=.3; x^2+e^(-x)"
%    0.830818220681717
