Benjamin Gompertz (1779-1865) was a British mathematician and pioneering
actuarial scientist, who overcame significant social barriers due to
antisemitism. We would all like to live forever, and actuaries are in
the business of telling us that we probably can't. Based on mortality
data, Gompertz constructed a model in which an initial population $N_\zu{o}$
of babies born at $t=0$ becomes at a later time $t$ a surviving population
\begin{equation*}
  N = N_\zu{o} e^{1-e^t} \qquad ,
\end{equation*}
where I've simplified the expression by leaving out some constants.
If you've survived to age $t$, then your probability of dying in the coming
year is
\begin{equation*}
  -\frac{\Delta N}{N} \qquad ,
\end{equation*}
where $-\Delta N$ is the number of deaths per year. Therefore the death rate is
\begin{equation*}
  -\frac{1}{N}\frac{\der N}{\der t} \qquad .
\end{equation*}
Show that in the Gompertz model, this death rate is proportional to
$e^t$. This exponential rate of increase is demonstrated in the figure.
