There are no kinks, endpoints, etc., so
extrema will occur only in places where the derivative is zero.
Applying the chain rule, we find the derivative to be $\cos(\sin(\sin x))\cos(\sin x)\cos x$.
This will be zero if any of the three factors is zero. We have $\cos u=0$ only when $|u| \ge \pi/2$,
and $\pi/2$ is greater than 1, so it's not possible for either of the first two factors to equal zero.
The derivative will therefore equal zero if and only if $\cos x=0$, which happens in the same places
where the derivative of $\sin x$ is zero, at $x=\pi/2+\pi n$, where $n$ is an integer.

\anonymousinlinefig{soln-sin-sin-sin}
