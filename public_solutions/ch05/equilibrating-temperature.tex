(a) Since $T$ has units of degrees, both terms on the right-hand side must also have
units of degrees. The first term on the right is $a$, so $a$ has units of degrees.
The second term consists of $b$ multiplied by an exponential. The exponential is
unitless, so $b$ must have units of degrees. The input to the exponential must be
unitless as well, so $c$ must have units of inverse seconds ($\sunit^{-1}$).\\
(b) $\der T/\der t=bce^{-ct}$\\
On the left side, the units are what is implied by the original interpretation of
the Leibniz notation: we have a small change in temperature divided by a small change
in time, so the units are degrees per second ($\degunit/\sunit$). On the right, the
units come from the factor $bc$, since the exponential is unitless. The units of
$bc$ are degrees multiplied by inverse seconds, $(\degunit)(\sunit^{-1})$, and this
matches what we had on the left-hand side.
(c) In this limit, the the temperature approaches $a$, and the derivative approaches zero.
It makes sense that the derivative goes to zero, since eventually the beer will be
in thermal equilibrium with the air.\\
(d) Physically, $a$ is the temperature of the air, $b$ is the difference in
temperature at $t=0$ between the air and the beer, and $c$ measures how good
the thermal contact is between the air and the beer --- e.g., if the beer is in
a styrofoam container, $c$ will be small.
