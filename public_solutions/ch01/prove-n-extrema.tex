Since polynomials don't have kinks or endpoints in their graphs, the maxima and minima must be points where the
derivative is zero. Differentiation bumps down all the powers of a polynomial by one, so 
the derivative of a third-order polynomial is a second-order polynomial. A second-order polynomial
can have at most two real roots (values of $t$ for which it equals zero), which are given by the
quadratic formula. (If the number inside the square root in the quadratic formula is zero or negative,
there could be less than two real roots.) That means a third-order polynomial can have at most two
maxima or minima.
