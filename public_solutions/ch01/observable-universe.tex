The function $v=(4/3)\pi(ct)^3$ looks scary and complicated, but it's nothing more than a constant
multiplied by $t^3$, if we rewrite it as $v=\left[(4/3)\pi c^3\right]t^3$. The whole thing
in square brackets is simply one big constant, which just comes along for the ride
when we differentiate. The result is $\dot{v}=\left[(4/3)\pi c^3\right](3t^2)$, or,
simplifying, $\dot{v}=\left(4\pi c^3\right)t^2$. (For further physical insight, we can
factor this as $\left[4\pi (ct)^2\right]c$, where $ct$ is the radius of the expanding sphere, and
the part in brackets is the sphere's surface area.)

For purposes of checking the units, we can ignore the unitless constant $4\pi$, which just
leaves $c^3t^2$. This has units of $(\text{meters per second})^3(\text{seconds})^2$, which
works out to be cubic meters per second. That makes sense, because it tells us how quickly
a volume is increasing over time.
