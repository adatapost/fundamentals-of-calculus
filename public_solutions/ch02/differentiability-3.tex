A cusp will occur if both branches are vertical at $x=0$, i.e.,
if $f'$ blows up there.

For positive values of $x$, the definition of $f$ is the same as
$x^p$, so by the power rule $f'=px^{p-1}$. For negative $x$,
the horizontal flip property of the derivative
(p.~\pageref{properties-of-derivative}) tells us that $f'$
equals minus the value of the derivative at the corresponding point
on the right.

For $p<1$, the derivative blows up, and $f$ has a cusp.

If $f$ is to be differentiable at $x=0$, then it can't have a kink.
By the symmetry property described above, this requires that $f'(0)=0$.
This occurs if $p>1$. The function is nondifferentiable when $p \le 1$.
