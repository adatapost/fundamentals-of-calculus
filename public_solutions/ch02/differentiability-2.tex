The function $f(x)=1/\sin x$ can be written as a composition $f(x)=g(h(x))$ of the
functions $g(x)=1/x$ and $h(x)=\sin x$. We don't have to recall anything about
the sine function, $h$, except that it looks like a sine wave, so that it's clearly
continuous and differentiable everywhere. The function $g$, on the other hand,
is discontinuous at $0$, so it will be discontinuous at any $x$ such that
$\sin x=0$, and $f$ will also be discontinuous in these places. The relevant
values of $x$ are $\{\ldots, -2\pi, -\pi, 0, \pi, 2\pi, \ldots\}$.
Since $f$ is discontinuous at these points, it is also nondifferentiable there,
because discontinuity implies nondifferentiability.
