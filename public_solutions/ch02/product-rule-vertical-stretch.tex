The vertical stretch rule says that
stretching a function $y(x)$ vertically to form a new function $ry(x)$ multiplies its
                   derivative by $r$ at the corresponding points.
That is, if $r$ is a constant, then $(ry)'=ry'$. To prove this using the product
rule, we have
\begin{equation*}
  (ry)' = r'y+y'r \qquad .
\end{equation*}
But if $r'$ is a constant, then $r'=0$, so the first term is zero, and we have the
claimed result.
